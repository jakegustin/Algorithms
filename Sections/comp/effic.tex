\section{Manipulating Numbers in Base 2}

\label{sec:effic}

\begin{theo}[Binary Length of a Number - $\|a\|$]
    
    The binary length of an integer $a_{10}$ in binary representation, is given by:
    
    \[
    \|a\| := 
    \begin{cases} 
    \lfloor \log_2 |a| \rfloor + 1 & \text{if } a \neq 0, \\
    1 & \text{if } a = 0,
    \end{cases}
    \]
    
    \noindent
    as $\lfloor \log_2 |a| \rfloor + 1$ correlates to the highest power of 2 required to represent $a$.
\end{theo}

\begin{Example}[Base 10 and Base 2 Lengths]

\label{ex:log_length}

\vspace{-1em}
Think of base 10 and a 9 digit number $d=684,301,739$.
To reach 9 digits takes $10^8$; The exponent plus 1 yields $\|d\|$. Hence, $\lfloor \log_{10} d\rfloor+1$ is $\|d\|$.
Now, let there be a 7 digit binary number $b=1001000$, which expanded is:

$$ (1\cdot 2^6) + (0\cdot 2^5) + (0\cdot 2^4) + (1\cdot 2^3) + (0\cdot 2^2) + (0\cdot 2^1) + (0\cdot 2^0) = 72,$$
\noindent
Taking $6$ powers of 2 to reach $72$, we add 1 to get $\|b\| = 7$. Hence, $\|b\| = \lfloor \log_2 b \rfloor + 1$.

\noindent
Additionally, if $a=0_2$ then $\|a\|=1$. as $a^0=1$.
\end{Example}


\begin{theo}[Splitting Higher and Lower Bits]

    Let $a$ be a binary number with $n$ bits. We can split $a$ into two numbers $A_1$ and $A_0$ with $n/2$ bits each,
    representing the first and second halves respectively. Where:
    
        $$A_1:=\frac{a}{2^{\ceil{n/2}}} \quad \text{ and } \quad A_0:=a \text{ mod } 2^{\ceil{n/2}}$$

\end{theo}

\begin{Example}[Splitting Base 10]

\label{ex:split_base}

\vspace{-1em}
Start with base 10, and split the first 5 digits of $a=745,562,010$.
we take the length $\|a\|:=\floor{\log_{10}(745,562,010)}+1=9$, as $10^8\leq 745,562,010<10^9$:
\[ A_1=\frac{745,562,010}{10^{\ceil{9/2}}}=7455, \quad \text{ and } \quad A_0=745,562,010 \text{ mod } 10^{\ceil{9/2}}=62,010 \]
\noindent
We can use the same bit shifting technique from Theorem (\ref{theo:bit_shift}) for base 10:
$$[745,562,010]_{10} \text{ right shift by 5, } [000,007,455]_{10}\ 62,010.$$ 
\end{Example}

\begin{Example}[Splitting Base 2]

\label{ex:split_base2}

\vspace{-1em}
Continuing from Example (\ref{ex:split_base}), we can also split the same number in base 2.\\
Given $[1111\ 1111\ 1001\ 1001_2]$ ($32665_{10}$), which is 16 bits long, we can split it in half, as follows:

\[ 
A_1=\frac{1111\ 1111\ 1001\ 1001_2}{2^{\ceil{16/2}}}=1111\ 1111_2
\]
\[A_0=1111\ 1111\ 1001\ 1001_2 \text{ mod } 2^{\ceil{16/2}}=1001\ 1001_2 \]

\vspace{-2em}
\end{Example}


\noindent


